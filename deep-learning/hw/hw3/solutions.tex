
\documentclass{article}
\usepackage[utf8]{inputenc}
\usepackage{braket}
\usepackage{amsmath}
\setlength{\parskip}{\baselineskip}%
\setlength{\parindent}{0pt}%
\usepackage{graphicx}

\begin{document}


\section*{Question 1}

(5 points) From what you have learnt about general attention mechanism, explain how and when would it
become self-attention.
(5 points) Similarly, please explain how and when attention would become cross-attention.
(10 points) For each of the above attention mechanisms, please find a research paper that utilize it. Please
(a) provide a link to the paper, then (b) shortly describe the reason why they use such attention mechanism
in their work


\section*{Question 2}

Attention mechanisms can be categorized into soft-attention and hard-attention. Each of the mechanisms
have both pros and cons. However, soft-attention is much more popular than hard-attention.
(10 points) Please find and declare pros and cons of each attention mechanism.
(5 points) Give an example of a task in which hard-attention is superior to soft-attention and vice versa.
(5 points) Explain why soft-attention is recognized by the community more than hard-attention.


\section*{Question 3}

(10 points) Autoencoder (AE) is an popular neural network scheme that is trained in unsupervised manner
for various applications, e.g., dimensionality reduction, image denoising, image compression, etc. Variational
Autoencoder (VAE) is firstly introduced in 2014. Please explain the advantages of VAE over AE.
(10 points) Please list 2 applications of VAE and briefly describe how it is used (trained and deployed) in
each application



\end{document}