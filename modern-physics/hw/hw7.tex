\documentclass[12pt]{article}
\usepackage[utf8]{inputenc}
\usepackage{amsmath}
\usepackage{graphicx}
\setlength{\parskip}{\baselineskip}%
\setlength{\parindent}{0pt}%



\title{Modern Physics Homework 7}

\author{Tyler Tracy}

\begin{document}

\maketitle

\section*{Chapter 8, Problem 7}


(a) What is the electronic configuration of Fe?

[Ar] 3$d^6$ 4$s^2$


(b) In its ground state, what is the maximum possible total $m_s$ of its electrons?

Argon has an $m_s$ of 0. The 3d shell can hold 10 electrons, so 5 of the electrons could have an $m_s$ of $\frac{1}{2}$ and the other electron could have an $m_s$ of $-\cfrac{1}{2}$. The filled s shell has an $m_s$ of 0. So that means the total $m_s$ is $ 5 * \frac{1}{2} - \frac{1}{2} = 2$


(c) When the electrons have their maximum possible total $m_s$, what is the maximum total $m_l$?

The first 5 electrons have to take the $m_l$ values of $-2, -1, 0, 1, 2$ Then the last electron should take the largest possible value of $m_l$ which is 2. Which would be the max possible $m_l$.

(d) Suppose one of the d electrons is excited to the next highest level. What is the maximum possible total $m_s$, and when ms has its maximum total what is the maximum total $m_l$?

That would mean the electron moves to the 4d shell. So the $m_s$ in the 3d shell would be $5 * \cfrac{1}{2}$ and the $m_l$ would be 0 since all values are filled. And the 4d electron would take $m_s$ of $\cfrac{1}{2}$ and $m_l$ of 2.
So the max $m_s$ is $3$ and the max $m_l$ is $4$.


\section*{Chapter 8, Problem 8}


The ground state of singly ionized lithium (Z = 3) is 1$s^2$. Use the electron screening model to predict the energies of the 1$s^1$ 2$p^1$ and 1$s^1$ 3$d^1$ excited states in singly ionized lithium. Compare your predictions with the measured energies (respectively -13.4 eV and -6.0 eV).

$$ E_n = (-13.6 eV) \frac{Z^2_{eff}}{n^2} $$
$$ E_n = (-13.6 eV) \frac{2^2}{2^2} = - 13.6 eV$$
$$ E_n = (-13.6 eV) \frac{2^2}{3^2} = - 6.04 eV$$

These are slightly off from the predicted value. The 1$s^2$ has a small penetrating value that is making the energies actually a bit less negative.


\section*{Chapter 8, Problem 12}


A certain element emits a ${K_\alpha}$ X ray of wavelength 0.1940 nm. Identify the element.

$$ \Delta E = (10.2 eV)(Z - 1)^2 $$
$$ Z = \sqrt{\frac{\Delta E}{10.2 eV}} + 1 $$
$$ Z = \sqrt{\frac{hc}{\lambda * 10.2 eV}} + 1 $$
$$ Z = \sqrt{\frac{hc}{\lambda * 10.2 eV}} + 1 $$
$$ Z = 26 $$

The element is Iron.


\section*{Chapter 8, Problem 14}


Chromium has the electron configuration 4$s^1$ 3$d^5$ beyond the inert argon core. What are the ground-state L and S values?

We have 5 electrons in the 3d shell. Each can have a spin value of 1/2. So the total spin is 5/2. Thus $S = 5/2$. We are forced to give the values (2, 1, 0, -1, -2) to the electron fro its $m_l$. So the total $m_l$ is 0. Thus $L = 0$.



\section*{Chapter 8, Problem 15}

Use Hund's rules to find the ground-state L and S of

(a) Ce, configuration [Xe]6$s^2$ 4$f^1$ 5$d^1$

We give all of the 5d electrons $+1/2$ spin. So the total spin of them is $ 5 / 2$. We give the 4f electron a spin of $1/2$ thus the total spin is $ S = 3$

The 5 electrons in the 5d shell have to take the values (2, 1, 0, -1, -2). So the total $m_l$ is 0. The 4f electron can take the maximum L values of 3. So the total $m_l$ is 3. Thus $L = 3$

(b) Gd, configuration [Xe]6$s^2$ 4$f^7$ 5$d^1$

We give all of the 5d electrons $+1/2$ spin. So the total spin of them is $ 5 / 2$. There are 14 electrons in the 4f shell. So we give 7 of them $+1/2$ spin. Thus the total spin is $ S = 7/ 2 + 5 / 2 = 6$

Since both the 5d shell and the 4f shell are half full the total $m_l$ is 0. Thus $L = 0$

(c) Pt, configuration [Xe]6$s^1$ 4$f^{14}$ 5$d^9$

The 6s and 4f shell are full so they don't effect the values. We give 5 of the electrons in the 5d shell a spin of $+1/2$ and 4 of them a spin of $-1/2$. Thus the total spin is $ S = 5/2 + -4/2 = 1/2$ We can give the first 5 electrons $m_l$ values of (2, 1, 0, -1, -2). The other 4 electrons can take values of (2,1,0, -1). So the total $m_l$ is 2. Thus $L = 2$



\end{document}