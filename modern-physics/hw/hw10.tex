\documentclass[12pt]{article}
\usepackage[utf8]{inputenc}
\usepackage{amsmath}
\usepackage{graphicx}
\setlength{\parskip}{\baselineskip}%
\setlength{\parindent}{0pt}%

\title{Modern Physics Homework 10}

\author{Tyler Tracy}

\begin{document}

\maketitle


Chapter 13, Problems, 20, 22 and 25.


\section*{Chapter 13, Problem 20}

Find the Q value (and therefore the energy released) in the fission reaction ${}^{235}U + n \rightarrow {}^{93}Rb + {}^{141}Cs + 2n.$ Use m(${}^{93}$Rb) = 92.922042 u and m(${}^{141}$Cs) = 140.920046u.

$$ Q = (m({}^{235}U)+ m(n) - m({}^{93}Rb) - m({}^{141}Cs) - 2m(n))c^2 $$
$$ Q = (235.043 + 1.00866 - 92.922042 - 140.920046 - 2(1.00866))(931.5 MeV/u) $$
$$ Q = 179 Mev $$


\section*{Chapter 13, Problem 22}

Show that the D-T fusion reaction releases 17.6 MeV of energy.

Equation:
$$ {}^{3}H + {}^{2}H \rightarrow {}^{4}He + n $$

$$ Q = (m({}^{3}H) + m({}^{2}H ) - m({}^{4}He ) - m(n))c^2 $$
$$ Q = (3.016 + 2.014 - 4.002602 - 1.00866)(931.5 MeV/u) $$
$$ Q = 17.6 MeV $$

\section*{Chapter 13, Problem 25}

Find the energy released when three alpha particles combine to form 12 C.

$$ {}^{4}He + {}^{4}He + {}^{4}He \rightarrow {}^{12}C $$

$$ Q = (3(m({}^4He)) - m({}^{12}C))c^2 $$
$$ Q = (3 * 4.00150617 - 12.011)(931.5 MeV/u) $$
$$ Q = - 6.03 MeV$$




\end{document}