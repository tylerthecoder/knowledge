\documentclass[12pt]{article}
\usepackage[utf8]{inputenc}
\usepackage{amsmath}
\usepackage{graphicx}
\setlength{\parskip}{\baselineskip}%
\setlength{\parindent}{0pt}%

\title{Modern Physics Homework 10}

\author{Tyler Tracy}

\begin{document}

\maketitle

\section*{Chapter 15, Problem 2}

The light from a certain galaxy is red-shifted so that the wavelength of one of its characteristic spectral lines is doubled. Assuming the validity of Hubble's law, calculate the distance to this galaxy.

\section*{Chapter 15, Problem 7}

A satellite is in orbit at an altitude of 150 km. We wish to communicate with it using a radio signal of frequency 109 Hz. What is the gravitational change in frequency between a ground station and the satellite? (Assume g doesn't change appreciably.)

\section*{Chapter 15, Problem 17}

Suppose the difference between matter and antimatter in the early universe were 1 part in 108 instead of 1 part in 109.  (a) Evaluate the temperature at which deuterium begins to form. (b) At what age does this occur? (c) Evaluate the temperature and the corresponding time of radiation decoupling (when hydrogen atoms form).

\section*{Chapter 15, Problem 18}

What was the age of the universe when the nucleons consisted of 60\% protons and 40\% neutrons?

\section*{Chapter 15, Problem 22}

Photons of visible light have energies between about 2 and 3 eV.
(a) Compute the number density of photons from the 2.73-K background radiation in that interval. (It is sufficient to characterize the visible region as E = 2.5 eV with dE = 1.0 eV.)
(b) Assume the eye can detect about 100 photons/cm3 . At what temperature would the back- ground radiation be visible? At what age of the universe would this have occurred?


\end{document}