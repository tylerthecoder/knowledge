\documentclass{article}
\usepackage[utf8]{inputenc}
\usepackage{braket}
\usepackage{amsmath}

\title{Modern Physics Homework 4}
\author{Tyler Tracy}

\begin{document}

\maketitle


\section{Chapter 6, Problem 1}

Electrons in atoms are known to have kinetic energies in the range of a few eV. Show that the uncertainty principle allows electrons of this energy to be confined in a region the size of an atom (0.1 nm)


\textbf{Solution:}

The electron's position is known to be the size of $ \Delta x = 0.1 nm $. With the uncertainty relation we can find what the uncertainty in the electron's momentum is.

$$\Delta x \Delta p \geq \frac{\hbar}{2}$$
$$ \Delta p \geq \frac{\hbar}{2 \Delta x}$$
$$ \Delta p \geq \frac{\hbar}{2 \times 0.1 nm}$$
$$ \Delta p \geq \frac{\hbar}{2 \times 0.1 nm}$$
$$ \Delta p \geq 5.27 \times 10^{-25} kg \cdot m/s$$

We can find the max kinetic energy of the electron by using the equation $E = \frac{p^2}{2m}$

$$\Delta KE = \frac{(\Delta p)^2}{2m}$$
$$\Delta KE = \frac{(5.27 \times 10^{-25} kg \cdot m/s)^2}{2m_e}$$
$$\Delta KE = 1.53 \times 10^{-19} J = .9535 eV$$

So we can see that the electron can have its kinetic energy known in the range of a few eV.


\section{Chapter 6, Problem 18}
In the n = 3 state of hydrogen, find the electron’s velocity, kinetic energy, and potential energy

\textbf{Solution:}

First we will find the bohr radius of the n = 3 state.

$$r_{3} = \frac{4\pi \hbar^2}{m_e e^2}$$
$$r_{3} = .4761nm $$

Now we find the Electrons kinetic energy and potential energy.

$$KE = \frac{1}{8 \pi \epsilon_0} \frac{e^2}{r} = 2.42 \times 10^{-19} J $$
$$U = -\frac{1}{4 \pi \epsilon_0} \frac{e^2}{r} = -4.84 \times 10^{-19} J $$

Now we find the velocity of the electron.

$$ v = \sqrt{\frac{2KE}{m_e}} = 729000 m/s $$

\section{Chapter 6, Problem 21}
An electron is in the n = 5 state of hydrogen. To what states can the electron make transitions, and what are the energies of the emitted radiations?

\textbf{Solution:}

An electron can jump to any state with a lower energy. The electron can jump to the n = 4, n = 3, n = 2, and n = 1 states. The energy of the emitted radiation is the difference in energy between the two states.

$$E_{5} = -\frac{13.6}{5^2} = -.54 eV$$

A jump to the n = 4 state would emit a photon with the difference in energy between the two states.

$$E_{4} = -\frac{13.6}{4^2} = -.85 eV$$

$$\Delta E = E_{5} - E_{4} = .31 eV$$

The same process can be done for the other states.

$$ E_{5\rightarrow3} = E_{5} - E_{3} = .97 eV$$

$$ E_{5\rightarrow2} = E_{5} - E_{2} = 2.86 eV$$

$$ E_{5\rightarrow1} = E_{5} - E_{1} = 13.1 eV$$




\section{Chapter 6, Problem 29}
An alternative development of the Bohr theory begins by assuming that the stationary states are those for which the circumference of the orbit is an integral number of de Broglie wavelengths. 

(a) Show that this condition leads to standing de Broglie waves around the orbit. 

(b) Show that this condition gives the angular momentum condition, Eq. 6.26, used in the Bohr theory.

\textbf{Solution:}

This condition implies that $2 \pi r = n \lambda$. Stating that the circumfernce can be traveled with some integer multile lengths of the wave.

If we use the de Broglie wavelength we get $2 \pi r = \frac{hn}{p}$

If we rearrange this equation we get $ \frac{nh}{2 \pi} = pr $

Which can be rewritten as $ n \hbar = mvr $.












\end{document}
