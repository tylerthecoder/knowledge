
\documentclass[12pt]{article}
\usepackage[utf8]{inputenc}
\usepackage{amsmath}
\usepackage{graphicx}
\usepackage{listings}
\usepackage{enumitem}
\usepackage{tcolorbox}
\usepackage{braket}
\usepackage{enumitem}

\title{Boolean circuit complexity}

\author{Tyler Tracy}

\begin{document}


Podolskii's first area of study he mentioned was boolean circuit complexity. A boolean function is a function $f$ that maps a set of zeros and ones to a single value. Represented as $f: \{0,1\}^n \rightarrow \{0,1\}$. Many different boolean functions are used in normal circuits. Podolskii was interested in trying to lower bounds for solving circuits. If a super polynomial lower bound was found that could imply that $P \neq NP$. There are different ways to constrain the problem to make it "easier" to find a lower bound like having the circuits be monotone or have a bounded depth but so far the results of them have not helped solve the general case. Threshold gates are a generalization of $AND$, $OR$, and $MAJ$ gates and Podolskii was able to show that there is a known lower bound for some constrained circuits with just $THR$ gates. Podolskii also researches ontology-mediated queries. These are similar to database queries but have logical statements or theorems that apply to the data. Executing some query on unbounded theorems in the general case is undecidable. But there are techniques for rewriting the query using the formulas to make it more manageable. The rewritten formula grows exponentially in the size of the query so it didn't seem practical. But people are using ontological databases in practice with limited queries and ontologies. Podolskii was able to identify a connection between these ontology-mediated queries and the boolean circuit complexity. Overall I enjoyed Podolskii's talk. I liked how theoretical his work is and how he is framing his work in terms of slowly chipping away at the $P = NP$ problem.




\end{document}