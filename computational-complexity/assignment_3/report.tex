\documentclass[12pt]{article}
\usepackage[utf8]{inputenc}
\usepackage{amsmath}
\usepackage{graphicx}
\usepackage{listings}
\usepackage{enumitem}
\usepackage{tcolorbox}
\usepackage{braket}
\usepackage{enumitem}

\title{Computational Complexity Assignment 3}

\author{Tyler Tracy}

\newtcolorbox{questionbox}{
        colframe=cyan!20!white,
        colback =cyan!20!white,
        top=0mm, bottom=0mm, left=0mm, right=0mm,
        arc=0mm,
%
        fontupper=\color{blue!70!black},
        fonttitle=\bfseries\color{blue!70!black},
        title=Question:
                        }
\setlength{\parskip}{\baselineskip}%
\setlength{\parindent}{0pt}%

\begin{document}

\maketitle

\section*{Problem 1}

\begin{questionbox}
	Let $CNF_k = \{ \braket{\phi} \mid \phi$ is a satisfiable CNF-formula where each variable appear in at most $k$ places $\}$.
	\begin{enumerate}[label=(\alph*)]
		\item Show that $CNF_2 \in P$.
		\item Show that $CNF_3$ is NP\_complete.
	\end{enumerate}
\end{questionbox}


To answer part a I would try to find an algorithm to figure it out

\subsection*{Part a}

There are $V$ variables and $C$ clauses.

I'll define a algorithm that will decide if some CNF formula is in $CNF_k$. It will use a recursive algorithm.


Take in a formula $\phi$ and first check that no variable appears more than 2 times. This should be possible in a couple passes over the formula. If it does then $REJECT$. Next find the smallest clause in $\phi$. We set the first variable to be satisfied. I.E True if it is a $x$ and False if it is a $\overline{x}$. Call the clause this variable is it $k_0$. Find the clause that contains the other occurrence of the variable and call it $k_1$. Note that the number of vars in $k_0$ is less than or equal to the number of vars in $k_1$.

We handle multiple cases to determine what to do

\begin{enumerate}
	\item If $k_0$ has a single variable and $k_1$ has a single variable and they are different signs (i.e. one is $\overline{x}$ and the other is $x$) then $REJECT$ this is not satisfiable.
	\item If the two instance of the variable are the same sign, then remove both clauses from $\phi$ and recurse on the new formula.
	\item If the two instance of the variable are different signs, then remove $k_0$ from $\phi$ and remove the other variable from $k_1$ and recurse on the new formula.
	\item If there is not another occurrence of the variable in the formula then remove $k_0$ from $\phi$ and recurse on the new formula.

\end{enumerate}

If we follow these rules we will either $REJECT$ or get a formula with no clauses, meaning we $ACCEPT$.

This does some work for each clause in the formula. There can be at most $2V$ clauses in the formula. The work consists of finding the other occurrence and removing other variables from the formula, which would take no more than $O(V^2)$ work. So the total work is at most $O(V^3)$ which is polynomial in the size of the formula, so it is in $P$.

I give an example of this formula working

$$ (x \lor \overline{y} \lor z) \land (\overline{z} \lor \overline{w}\lor a ) \land (\overline{a}) \land (w) \land (k \lor \overline{x}) $$
The $(\overline{a})$ clause is the smallest, so we set $a$ to False and continue.
$$ (x \lor \overline{y} \lor z) \land (\overline{z} \lor \overline{w}) \land (w) \land (k \lor \overline{x}) $$
The $(w)$ clause is the smallest, so we set $w$ to True and continue.
$$ (x \lor \overline{y} \lor z) \land (\overline{z})  \land (k \lor \overline{x}) $$
The $(\overline{z})$ clause is the smallest, so we set $z$ to False and continue.
$$ (x \lor \overline{y}) \land (k \lor \overline{x}) $$
The $(x \lor \overline{y})$ clause is the smallest, so we set $x$ to True and continue.
$$ (k) $$
Set $k$ to True and we are done.


This process will always work because if you handle the smallest clause first, then you will have enough control to satisfy the formula.



\subsection*{Part b}

To show that something is $NP$-complete, we need to show that it is in $NP$ and is $NP$-Hard

$CNF_3$ is in $NP$ because we can use a certificate of a satisfiable variable assignment to decide if a formula is satisfiable. Having all of the truth values makes it very easy to check in polynomial time. The certificate could be a string of 0s and 1s where each position corresponds to a variable. If the value is 0 then the variable is false and if the value is 1 then the variable is true. Using this to check all the clauses should be easily obtainable in less than $O(n^2)$ time.

To prove that $CNF_3$ is $NP$-Hard I will use a reduction from $3SAT$.

We have a formula $\phi$ in $3SAT$ with $n$ variables and $m$ clauses. For each variable $x$ we will create a new variable $x_i$ for each occurrence of $x$ incrementing $i$ each time. Replace each $x$ in $\phi$ with the corresponding $x_i$ value. We now need some mechanism to make sure all the $x_i$s have the same value so they act as one variable. We can do this by adding clauses like the following $ (x_0 \lor \overline{x_1}) \land (x_i \lor \overline{x_2}) \land \dots \land (x_{i-1} \lor \overline{x_i}) \land (x_i \lor \overline{x_{0}})$ This ensures that all variables are the same value, thus acting as a single variable. This gadget uses each variable twice and then each variable is used once in the actual formula so no variable is used more than 3 times.

We need to show that this formula ($\phi_0$) is satisfiable if and only if $\phi$ is satisfiable.

If $\phi$ is satisfiable, then the main part of $\phi_0$ is satisfiable since it is the same as $\phi$. The gadget will also be satisfiable since it is satisfiable iff the variables $x_i$ are the same, which they will have to be since they are all assigned the same value as $x$ in the original formula.

If $\phi_0$ is satisfiable, then $\phi$ is satisfiable. The gadget forces each $x_i$ to be the same so we can just remove the gadget and use the main part of $\phi_0$ as $\phi$.

Since this condition holds both ways $CNF_3$ is $NP$-Complete


\section*{Problem 2}

\begin{questionbox}
	Let $L_1 = \{ \braket{M,c} \mid M$ is a Turing machine which decides its language in c steps $\}$, (i.e. in c
	steps for all inputs, which is $O(1)$ time). Prove that $L_1$ is decidable. [HINT: recall that this
	must hold for all inputs.]
\end{questionbox}

To show that $L_1$ is decidable I'll define a new machine $D$ that will decide $L_1$.

$D$ takes as input $M$ and $c$. It will run $M$ on all inputs of size $c+1$ or less. If $M$ takes more than $c$ steps, then $D$ will reject. If $M$ takes less than $c$ steps, then $D$ will accept. It doesn't need to worry about inputs of size greater than $c+1$ because $M$ can't process those inputs of $c$ or greater in less than $c$ steps. We check up to inputs $c+1$ to ensure that M will not accept any inputs of size $c+1$ or greater.

\section*{Problem 3}

\begin{questionbox}
	Let $L_2 = \{ \braket{M} \mid M $ is a Turing machine which decides its language in $O(1)$ steps $\}$. Prove that
	$L_2$ is undecidable. [Hint: you may want to try using a reduction from HALT.]
\end{questionbox}


$L_2$ is the set of turing machines that finish in some constant number of steps. I will show this is undecidable by reducing from $HALT$. If there exists some polynomial time reduction from $HALT$ to $L_2$, that means a machine that can decide $L_2$ can be used to help solve $HALT$ in polynomial time.

I'll define a machine $D$ that will decide $L_2$ and use this machine to solve $HATL$. The reduction is rather simple. If $D$ accepts, then $HALT$ accepts. If $D$ rejects, then $HALT$ rejects. This is because $D$ will accept iff the machine it is given finishes in constant time for all inputs, meaning that it will always halt.

Since a solution to $L_2$ can be used to solve $HALT$ in polynomial time, $L_2$ is undecidable.

\end{document}
