\documentclass{article}
\usepackage[utf8]{inputenc}
\usepackage{braket}
\usepackage{amsmath}

\title{Particle Physics}
\author{tylertracy1999 }
\date{November 2021}

\begin{document}

\maketitle




\section{Basic Equations}

$$ \omega = 2 \pi f $$
$$n\lambda = 2 \pi r$$
$$ \lambda = \frac{2\pi}{k} $$
$$ p = \hbar k $$
$$ E = \frac{p^2}{2m} $$
$$ E =  \hbar \omega $$

$\lambda$ is particle wave length
K is equivalent to particle momentum. It is quantized.

\section{Wave Equations}

Simple Wave Equation
$$ \psi = e^{i(kx - \omega t)}$$

Another Wave Equation
$$ \frac{d\Psi}{dt} = -v \frac{d\Psi}{dx}  $$

In particle scattering a particle hits a barrier and it is annihilated and then recreated somewhere else.

Represented by:
$\Psi^+ \Psi \ket{k_i} \rightarrow \ket{k_i}$

You integrate over all of time because you don't know when the scattering will happen.


Particle Decay
$\Psi^+ \Psi^+ \Psi \ket{k_1} \rightarrow \ket{k_2 k_3}$

Integrate over all of space because we don't know where the particle will decay.



Dirac postulated negative values for $\omega$. Which have negative energy.


Dirac Equation (Derived from Right $\Psi$ and Left $\Psi$)




\section{Special Relativity}

$E = \sqrt{p^2c^2 + m^2c^4}$

Let $c = 1$

$E = \sqrt{p^2 + m^2}$

Let $ \hbar = 1$

$ E = \sqrt{k^2 + m^2} $

$ \omega^2 = k^2 + m^2 $

From Dirac
$ \omega^2 = \alpha^2 k^2$

So $\alpha^2$ probably equals 1

$\omega = \alpha k + \beta m$

From this we get that $\alpha^2 = 1$ $\beta^2 = 1$ and $\alpha \beta + \beta \alpha = 0$

% $$\alpha = $$\begin{pmatrix}
%   1 & 0\\
%   0 & -1
% \end{pmatrix}

% $\beta = $\begin{pmatrix}
%   0 & 1\\
%   1 & 0
% \end{pmatrix}

\section{Examples}

Two Types of particles
- Fermions
- Bosons

Only Fermions have the Pauli Exclusion Principle. Meaning no two particles can have the same energy level. Thus each fermion has a different value for k.






\end{document}
