\documentclass{article}
\usepackage[utf8]{inputenc}
\usepackage{braket}
\usepackage{amsmath}

\title{Modern Physics Homework 1}
\author{Tyelr Tracy}

\begin{document}

\maketitle


\section*{Q 3}

The lowest energy of a particle in an infinite one-dimensional well is 4.4 eV. If the width of the well is doubled, what is its lowest energy?

\textbf{Solution:}

The equation for The energy of a particle in a 1 dimensional well is given by:

$$ E = n^2\cfrac{h^2}{8mL^2} $$

$h$ is Planck's constant, $m$ is the mass of the particle, and $L$ is the length of the well.

So if we double the length of the well then the energy would go down by a fourth

$$ E = n^2\cfrac{h^2}{8m(2L)^2} $$
$$ E = n^2\cfrac{h^2}{8m4L^2} $$
$$ E = (\cfrac{1}{4})n^2\cfrac{h^2}{8mL^2} $$
$$ E = \cfrac{1}{4}E_0 $$
$$ E = \cfrac{1}{4} {4.4ev} $$
$$ E = 1.1 eV $$


\section*{Q 6}

What is the minimum energy of a neutron ($mc^2$ = 940 MeV) confined to a region of space of nuclear dimensions $(1.0 x 10^{-14} m)?$

\textbf{Solution:}

The equation for The energy of a particle in a 1 dimensional well is given by:

$$ E = n^2\cfrac{h^2}{8mL^2} $$

We know that L = $10^{-14} m$ and $m = \cfrac{940 MeV}{c^2}$ and $ n = 1 $

Plugging in we get:

$$ E = \cfrac{h^2}{8(940 MeV/c^2)(10^{-14} m)^2} $$

$$ E = 495958 J = 2 MeV $$


\section*{Q 14}

A particle is trapped in an infinite one-dimensional well of width L. If the particle is in its ground state, evaluate the probability to find the particle

(a) between x = 0 and x = L/3;

\textbf{Solution:}

The probability equation is given by

$$ P = \int_{0}^{L/3} |\psi(x)|^2 dx $$
$$ P = \int_{0}^{L/3} \cfrac{2}{L} sin^2 \cfrac{ \pi x}{L} dx $$
$$ P = \cfrac{2}{L} \int_{0}^{L/3} sin^2 \cfrac{ \pi x}{L} dx $$
$$ u = \cfrac{\pi x}{L}, P = \cfrac{2}{\pi} \int_{0}^{\pi/3} sin^2 u dx $$
$$ P = \cfrac{2}{\pi} (\cfrac{u}{4} - \cfrac{sin(2u)}{4}) \rvert_{0}^{\pi / 3}  $$
$$ P = .195 $$



(b) between x = L/3 and x = 2L/3;

$$ P = \cfrac{2}{\pi} (\cfrac{u}{4} - \cfrac{sin(2u)}{4}) \rvert_{\cfrac{\pi}{3}}^{\cfrac{2\pi}{3}}  $$
$$ P = .609 $$


(c) between x = 2L/3 and x = L.

$$ P = \cfrac{2}{\pi} (\cfrac{u}{4} - \cfrac{sin(2u)}{4}) \rvert_{\cfrac{2\pi}{3}}^{\pi}  $$
$$ P = .195 $$


\end{document}

/drive/1KNBbRfu3qRhVETs5mJhVar9o08bLm1T0