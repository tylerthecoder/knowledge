\documentclass[12pt]{article}
\usepackage[utf8]{inputenc}
\usepackage{amsmath}
\usepackage{graphicx}
\setlength{\parskip}{\baselineskip}%
\setlength{\parindent}{0pt}%

\title{Modern Physics Homework 9}

\author{Tyler Tracy}

\begin{document}

\maketitle

\section*{Chapter 12, Problem 19}

Ordinary potassium contains 0.012 percent of the naturally occurring radioactive isotope ${}^{40}$K, which has a half-life of 1.3x$10^9$y.

(a) What is the activity of 1.0 kg of potassium?

$ a = \lambda N $
$$ N = 1kg K * 1000 \frac{g}{kg} * .00012 \frac{{}^{40}K}{K} * 6.022*10^{23} \frac{{}^{40}K}{mol} * \frac{mol}{40g}  $$
$$ N = 1.81 * 10^{21} {}^{40}K $$
$$ \lambda = \frac{0.693}{t_{1/2}} = \frac{0.693}{1.3*10^9} = 5.33 * 10^{-11}$$
$ a = 5.33 * 10^{11} * 1.81 * 10^{21} = 9.63*10^{10} $

(b) What would have been the fraction of ${}^{40}$ K in natural potassium 4.5x$10^9$y ago?

$ N = N_0 e^{-\lambda t} $
$ 1.81 * 10^{21} = N_0 e^{-5.33*10^{-11} * 4.5*10^9} $
$ N_0 = 2.301 * 10^{21} $

$ \frac{N_0}{\frac{N}{.00012}} = .00023 $

.023 percent

\section*{Chapter 12, Problem 25}
${}^{75}$Se decays by electron capture to ${}^{75}$As. Find the energy of the emitted neutrino.

Electron capture equation

$$ {}^{75}Se + e^{-} \rightarrow {}^{75}As$$

$$Q = (m({}^{75}Se) - m({}^{75}As))c^2$$

$$ Q= (74.9225229u - 74.921596u) * (931.5 MeV/u)$$

$$ Q = .8624 MeV $$

All energy goes to the emitted neutrino


\section*{Chapter 12, Problem 26}

${}^{15}O$ decays to ${}^{15}N$ by positron beta decay.

Equation

$$  $$

(a) What is the Q value for this decay?

$$ Q = [m({}^{15}O) - m({}^{15}N) - 2m_e]c^2$$

$$ Q = (15.003065u - 15.000109u - 2*0.00054858u)*(931.5 MeV/u) = 1.73MeV $$

(b) What is the maximum kinetic energy of the positrons?

The maximum energy is all of the energy so $ E_{max} = Q $


\section*{Chapter 12, Problem 28}

Compare the recoil energy of a nucleus of mass 200 that emits

(a) a 5.0-MeV alpha particle, and

$$ K = \frac{E_{\alpha}^2}{2Mc^2} $$
$$ K = \frac{m_\alpha^2*c^4}{2*Mc^2} $$
$$ K = \frac{m_\alpha*c^2}{2*M} $$
$$ K = \frac{(4u)^2*(931*10^9\frac{eV}{u})}{2*200u} $$
$$ K = 37.2 MeV $$

(b) a 5.0-MeV gamma ray.

$$ K = \frac{E_{\gamma}^2}{2Mc^2} $$
$$ K = \frac{5*10^{9 * 2}}{2 * 200 931*10^9} $$
$$ K = 67.1 eV $$

\section*{Chapter 12, Problem 30}

The radioactive decay of ${}^{232}$Th leads eventually to stable ${}^{208}$Pb. A certain rock is examined and found to contain 3.65 g of ${}^{232}Th$ and 0.75g of ${}^{208}Pb$. Assuming all of the Pb was produced in the decay of Th, what is the age of the rock?


$$ N_{pb} = .75g * 6.022*10^{23} \frac{P}{mol} * \frac{mol}{207.97g} = 2.172 * 10^{24} $$
$$ N_{th} = 3.65g * 6.022*10^{23} \frac{Th}{mol} * \frac{mol}{232.038g} = 9.47 * 10^{21} $$
$$ R = \frac{9.47 * 10^21}{2.172*10^24} = .0043 $$

$$ t_{1/2} = 1.42*10^10y $$

$$ t = \frac{t_{1/2}}{0.693}ln(\frac{1}{R} + 1) $$

$$ t= 1.117 * 10^{11} y$$







\end{document}