\documentclass{article}
\usepackage[utf8]{inputenc}
\usepackage{braket}
\usepackage{amsmath}

\title{Modern Physics Homework 1}
\author{Tyelr Tracy}

\begin{document}

\maketitle

Chapter 1, Q. 6
Chapter 2, Qs. 45, 49 and 53

\section*{Chapter 1, Q. 6}

Observer A, who is at rest in the laboratory, is studying a
particle that is moving through the laboratory at a speed of
0.624c and determines its lifetime to be 159 ns.

$v = 0.624c$, $t = 159 ns$, $c = 3*10^8 m/s$ \\

(a) Observer A places markers in the laboratory at the locations where
the particle is produced and where it decays. How far apart
are those markers in the laboratory? \\

\textbf{Solution:} \\
The observer is at rest to the laboratory so a classical kinematics equation works
$$ x = v * t $$
Substitute:
$$ x = 0.624c * 159 ns $$
Solve:
$$ x = (0.624 * 3 * 10^8) m/s * (159 * 10^{-9})s  $$
$$ x = 29.7 m  $$


(b) Observer B, who is traveling parallel to the particle at a speed of 0.624c,
observes the particle to be at rest and measures its lifetime to
be 124 ns. According to B, how far apart are the two markers
in the laboratory? \\

\textbf{Solution:}

$$ x = v * t $$
$$ x = 0.624c * 124 ns $$
$$ x = (0.624 * 3 * 10^8) m/s * (124 * 10^{-9})s  $$
$$ x = 23.2 m  $$


(c) Use the formula for length contraction to show that they are correctly measuring the distance between the markers in their frame of reference. \\

\textbf{Solution:}


$ \ell_0 =  29.7 $ This is the proper length because the this is the length measured in the frame of reference of the room. And the markers and stationary relative to the room.


A: $29.7m * \sqrt{1 - \frac{0}{c^2}} = 29.7m$

B: $29.7m * \sqrt{1 - \frac{.624c^2}{c^2}} = 23.2m$ \\


(d) Now explain to the two observers that the spacetime interval between two events is invariant. Name the two events they observe and calculate the spacetime interval between them according to each observer and show that they agree. \\

\textbf{Solution:}

Event 1: The particle passes the first marker

Event 2: The particle passes the second marker

To calculate the spacetime interval between the two events we use the equation:
$$ S = \sqrt{(c \Delta t)^2 - \Delta x^2} $$

Observer A calculates S to be:
$$ S = \sqrt{(3*10^8 m/s * 159ns)^2 - (29.7m)^2 } $$
$$ S =  37.3$$

Observer B calculates S to be:
$$ S = \sqrt{(3*10^8 m/s * 124ns)^2 - (0m)^2 } $$
$$ S = 37.3$$

S is the same for both observers


\section*{Chapter 2, Q 45}
Suppose we want to send an astronaut on a round trip to
visit a star that is 200 light-years distant and at rest with
respect to Earth. The life support systems on the spacecraft
enable the astronaut to survive at most 20 years.

(a) At what speed must the astronaut travel to make the round trip in
20 years of spacecraft time? \\

\textbf{Solution:}

Use the length contraction equation.

$t = 20 years$

$\ell_0 = 200 lightyears$

$c = 1 \cfrac{\text{lightyear}}{\text{year}} $

$\ell = v * t$

$\ell = \ell_0 \sqrt{1 - \frac{v^2}{c^2}}$

$vt = \ell_0 \sqrt{1 - v^2}$

Solve for v

$v = \sqrt{\cfrac{1}{\cfrac{t^2}{\ell_0^2} + 1}}$

Substitute

$v = \sqrt{\cfrac{1}{\cfrac{20^2}{200^2} + 1}}$

$v = 0.995 c$ \\


(b) How much time passes on Earth during the round trip? \\

\textbf{Solution:}

Use the time dilation equation.

$ \Delta t = 20 $ years

$\Delta t = \cfrac{\Delta t_0}{\sqrt{1-\cfrac{v^2}{c^2}}}$

Substitute

$20 years = \cfrac{\Delta t_0}{\sqrt{1-\cfrac{.995c^2}{c^2}}}$

Rearrange and solve

$ \Delta t_0 = 1.997 $ years



\section*{Chapter 2, Q 49}
Observer O sees a light turn on at x = 524 m when
t = 1.52 us. Observer O' is in motion at a speed of 0.563c
in the positive x direction. The two frames of reference are
synchronized so that their origins match up (x = x' = 0) at
t = t' = 0.

(a) At what time does the light turn on according to O'? \\

\textbf{Solution:}

Solve with the time dilation equation.

$t = 1.52 \mu s$, $v = .563c$

$\Delta t = \cfrac{\Delta t_0}{\sqrt{1-\cfrac{v^2}{c^2}}}$

$1.52 \mu s = \cfrac{\Delta t_0}{\sqrt{1-\cfrac{.563c^2}{c^2}}}$

$\Delta t_0 = 1.25 \mu s$ \\

(b) At what location does the light turn on in the reference frame of O'? \\

\textbf{Solution:}

Solve with length contraction equations

$\ell_0 = 524 m$

$\ell = \ell_0 \sqrt{1 - \frac{v^2}{c^2}}$

$\ell = 524m \sqrt{1 - \frac{.563c^2}{c^2}}$

$\ell = 433 m$


\section*{Chapter 2, Q 53}

\textbf{Solution:}

Electrons are accelerated to high speeds by a two-stage
machine. The first stage accelerates the electrons from rest
to v = 0.99c. The second stage accelerates the electrons
from 0.99c to 0.999c.

(a) How much energy does the first stage add to the electrons?


The amount of energy required to accelerate the electron is found by this equation:

$E = (\cfrac{1}{\sqrt{1 - \cfrac{v^2}{c^2}}} - 1)m_0c^2$

Substituting values we get

$E = (\cfrac{1}{\sqrt{1 - \cfrac{(.99c)^2}{c^2}}} - 1)m_ec^2$

Simplifying we get

$E = 6.089m_ec^2$

$ E = 4.985 \times 10^{-13} J$ \\[1em]



(b) How much energy does the second stage add in increasing the velocity by only 0.9

$ E_1 = $ energy from last problem, $v_2 = 0.999c$

Equation to solve:

$E = (\cfrac{m_ec^2}{\sqrt{1 - \cfrac{v^2}{c^2}}}) - E_1$

Substituting Variables

$E = (\cfrac{m_ec^2}{\sqrt{1 - \cfrac{(.999c)^2}{c^2}}}) - 4.985 \times 10^{-13} J$

Simplifying

$E = 1.8 \times 10^{-12} - 4.95 \times 10^{-13} $

$E = 1.31 \times 10^{-12} J$

\end{document}
