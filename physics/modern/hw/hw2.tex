\documentclass{article}
\usepackage[utf8]{inputenc}
\usepackage{braket}
\usepackage{amsmath}

\title{Modern Physics Homework 2}
\author{Tyelr Tracy}

\begin{document}

\maketitle

Chapter 2: Problems, 6, 13, 23, 25 and 47

\section*{Chapter 2, Q. 6}
An astronaut must journey to a distant planet, which is 200 light-years from Earth. What speed will be necessary if the astronaut wishes to age only 10 years during the round trip?

\section*{Chapter 2, Q. 13}
a) A physics professor claims in court that the reason he went through the red light ($\lambda$ = 650 nm) was that, due to his motion, the red color was Doppler shifted to green ($\lambda$ = 550 nm). How fast was he going?

b) to the problem that reads as follows b) assume another physicist witnessed the accident while travelling along the cross street (and thus transverse to the line of sight from them to the traffic light). They also claim that the light was red and not green. How fast was this physicist going?


\section*{Chapter 2, Q. 23}
Suppose Amelia traveled at a speed of 0.80c to a star that (according to Casper on Earth) is 8.0 light-years away.  Casper ages 20 years during Amelia's round trip. How much younger than Casper is Amelia when she returns to Earth?

\section*{Chapter 2, Q. 25}
(a) Using the relativistically correct final velocities for the collision shown in Figure 2.21a ($v^{'}_{1f} = -0.585c, v^{'}_{2f}=+0.294c$), show that relativistic kinetic energy is conserved according to observer O'.

(b) Using the relativistically correct final velocities for the collision shown in Figure 2.21b ($v_{1f}$ = -0.051c, $v_{2f}$ = +0.727c), show that relativistic kinetic energy is conserved according to observer O.

\section*{Chapter 2, Q. 47}
Observer O sees a red flash of light at the origin at t = 0 and a blue flash of light at x = 3.26 km at a time t = 7.63 us. What are the distance and the time interval between the flashes according to observer O', who moves relative to O in the direction of increasing x with a speed of 0.625c?  Assume that the origins of the two coordinate systems line up at t = t' = 0.



\end{document}
