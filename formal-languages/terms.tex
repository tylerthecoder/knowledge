\documentclass{article}
\usepackage[utf8]{inputenc}

\title{Formal Lang Assignment 1}
\author{Tyler Tracy}
\date{January 2020}

\begin{document}

\maketitle
1. The intersection is \{3\} \\
2. The union is \{1,2,3,a,b\} \\
3. A/B = \{1,2\} \\
4. B/A = \{a,b\} \\
5. Power set of A = \{\{1 \},\{2\},\{3\},\{1,2\},\{1,3\},\{2,3\},\{1,2,3\}\} \\
6. Power set of B = \{\{a\},\{b\},\{3\},\{a,b\},\{a,3\},\{b,3\},\{a,b,3\}\} \\

Truth Table for A implies B
\begin{displaymath}
\begin{array}{|c c|c|}
% |c c|c| means that there are three columns in the table and
% a vertical bar ’|’ will be printed on the left and right borders,
% and between the second and the third columns.
% The letter ’c’ means the value will be centered within the column,
% letter ’l’, left-aligned, and ’r’, right-aligned.
A & B & A \rightarrow B\\ % Use & to separate the columns
\hline % Put a horizontal line between the table header and the rest.
T & T & T\\
T & F & F\\
F & T & T\\
F & F & T\\
\end{array}
\end{displaymath}


Proof by induction: \\
(n+1)*n= 2*(1+2+...+n) \\
@ n = 1 : (1+1) * 1 = 2*(1) = 2 \\
@ n=k+1 : (k+2)*(k+1)=2*(1+2+...+k+k+1) \\
	    (k+2)*(k+1)=(k+1)*k+2*(k+1) \\
	    (k+2)*(k+1)=(k+2)*(k+1) \\

\end{document}
