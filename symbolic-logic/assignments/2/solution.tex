\documentclass[12pt]{article}

\usepackage{prooftrees,amsmath,turnstile}

\begin{document}

\section*{Solution to Problem 1}

\subsection*{Part (a)}

Argument: 
\begin{enumerate}
    \item $A \rightarrow [ (B \lor C) \rightarrow R]$
    \item $(R \lor S) \rightarrow T$
    \item $\therefore A \rightarrow (C \rightarrow T)$
\end{enumerate}

Truth tree: 

% \begin{prooftree}
%   {
%     to prove={}
%   }
%   [A \rightarrow ((B \lor C) \rightarrow R), checked, just={(premise)}, name=pr
%   [(R \lor S) \rightarrow T, checked, grouped, just={(premise)}
%   [\lnot (A \rightarrow (C \rightarrow T)) , checked, just={}
%   [A, checked, just={3}
%   [\lnot (C \rightarrow T), checked, just={3}, name=fa
%   [\tnot (\forall y) Fy, checked, grouped, just={}
%   [(\exists y) \tnot Fy, checked=b, just={}
%   [\tnot Fb, just={$\exists$D:!u}, name=nofb
%   [\tnot (\exists x) Fx, checked, just={}
%   [(\forall x) \tnot Fx, subs=a, just={}
%   [\tnot Fa, close={:fa,!c}, just={}
%                         ]
%                       ]
%                     ]
%                     [(\forall x) Fx, subs=b
%                       [Fb, close={:nofb,!c}, just={}, move by=2
%                       ]
%                     ]
%                   ]
%                 ]
%               ]
%             ]
%           ]
%         ]
%       ]
%     ]
%
% \end{prooftree}
%

\section*{Problem 2}


\section*{Problem 3}


\section*{Problem 4}


\subsection*{Part (a)}

\begin{enumerate}
    \item A sentence of form $p \lor q$ is a tautology if at least one of p and q is a tautology
    \item A sentence of form $p \lor q$ has truth table that always yeilds true, if at least one of p or q has a truth table that always yeilds true (by defintion of tautology)
    \item WLOG suppose p has a truth table that always yeilds true
    \item $p \lor q$ is always true since p is always true (by definition of logical or)
    \item This is true
\end{enumerate}

\subsection*{Part (b)}

Claim: For any set of sentence, $\Gamma$, if $\Gamma \vdash p \lor q$ then either $\Gamma \vdash p$ or $\Gamma \vdash q$.

This is false.

Suppose $\Gamma = \{p \lor q\}$. Then $\Gamma \vdash p \lor q$ but $\Gamma \nvdash p$ and $\Gamma \nvdash q$.


\subsection*{Part (c)}

Claim: If a set of sentences is inconsistent, then the set consisting of the negations of those sentences must be consistent. 

This is false.

Suppose $\Gamma = \{p, \lnot p\}$. Then $\Gamma$ is inconsistent but $\{\lnot p, p\}$ is also inconsistent.


\subsection*{Part (d)}

Claim: If the argument from the set of premises, $\Gamma$, to the conclusion ($A \land \lnot A$) is valid, then $\Gamma$ is inconsistent. 



\begin{itemize}
    \item If $\Gamma$ was consistent, then there would be a valid assignment of truth values that leds to all the sentences being true. 
    \item For a sentence to follow from $\Gamma$, it must be true in all of the truth assignments of $\Gamma$
    \item $A \land \lnot A$ is false in all truth assignments
    \item Therefore, $A \land \lnot A$ cannot follow from $\Gamma$
    \item We have a contradiction, so $\Gamma$ must be inconsistent
\end{itemize}

\section*{Problem 5}




\end{document}

