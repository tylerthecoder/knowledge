\documentclass[12pt]{article}
\usepackage[utf8]{inputenc}
\usepackage{amsthm}
\usepackage{amsmath}
\usepackage{amssymb}
\usepackage{graphicx}
\usepackage{listings}
\usepackage{enumitem}
\usepackage{tcolorbox}
\usepackage{braket}

\title{Tile Complexity Proofs}

\author{Tyler Tracy}

\setlength{\parindent}{0pt}%

\newtheorem{theorem}{Theorem}
\newtheorem{lemma}[theorem]{Lemma}

\begin{document}

\maketitle

This document aims to prove various results about tile complexity.

We will use the following definition throughout the proofs.

\begin{description}
    \item[Tile Assembly System (TAS):] A TAS is a tuple $\mathcal{T} = (T, \sigma, \tau)$ where $T$ is a finite set of tile types, $\sigma$ is the seed tile, and $\tau$ is the temperature.
    \item[$\alpha$:] An assembly that can be constructed by $\mathcal{T}$
    \item[{$\mathcal{A}[\mathcal{T}]$}:] The set of all assemblies that can be constructed from $\mathcal{T}$.
    \item[{$\mathcal{A}_{\square}[\mathcal{T}]$}:] The set of all terminal assemblies that can be constructed from $\mathcal{T}$.
    \item[$U_n$]: The set of unit vectors in n dimensions
    \item[{$TC(\mathcal{T})$}:] The tile complexity of a TAS $\mathcal{T}$.
    \item[$Dup(\alpha)$:] The set of duplicate tile types in an assembly $\alpha$.
    \item[$E(\alpha)$:] The set of connected edges in the assembly $\alpha$ 

    \item[diagonal cut]: A diagonal cut splits the unit directions into two groups that are diagonal of eachother. Like "cutting" a unit square with a diagonal line. 

        $D_1, D_2 \subset U_2$ are a diagonal cut if 

        \begin{enumerate}
            \item $D_1 \bigcup D_2 = U_2$
            \item $D_1 \bigcap D_2 = \emptyset$
            \item $\forall \vec{u} \in D_1 \vec{v} \in D_2 : |u - v| = 1$
        \end{enumerate}

    \item[Connected path:] In an assembly $\alpha$, a connected path is a sequence of tiles that are all adjacent and have positive glue strength between them.
    \item[Monotone assembly:] An assembly with a binding graph that only moves in two directions (forward and backwards). Starting with the seed tile, every tile only connects to another tile in the same two adjacent directions

        $\exists \vec{f}, \vec{b} \in U_2 : | \vec{f} - \vec{b} | = 1 \land \forall \vec{t} \in \text{ dom } \alpha, \{\vec{t},\vec{t}+\vec{f}\} \in E(\alpha) \implies \{\vec{t}, \vec{t}+\vec{b}\} \in E(\alpha)$

    \item[Non-branching:] An assembly where every tile connects to at most two other tiles. I.E., there are no branches

        $$ \forall \vec{t} \in \alpha |\{\vec{u} \in U : \vec{t} + \vec{u} \in \text { dom } \alpha | \leq 2 $$


\end{description}


Now we define the following lemmas that will be used in the proofs.

\begin{lemma}
    A non-branching monotone assembly is non-obstructive
\end{lemma}

\begin{proof}
    For an assembly to obstruct itself, there must be a location that can be reached multiple times. In a monotonic non-branching assembly, there is only one way to reach a location. Every tile only connects to the tile behind it and the tile in front of it. Furthermore, the tiles can never turn around and obstruct themselves because they are monotonic.
\end{proof}


\begin{lemma}[Pumping Lemma] 

    let $\mathcal{T}$ be a TAS

    $\exists \alpha \in \mathcal{A}_\square[\mathcal{T}] : \alpha$ is non-obstructive $\tau$-stable and $Dup(\alpha) \ne \emptyset \implies \exists \alpha_t \in \mathcal{A}_\square[\mathcal{T}] : |\alpha_t| = \infty$
\end{lemma}

\begin{proof}
    Proof by pumping. 

    Let $T_R$ be the tile type that is repeated. We can take the segment that is repeated and splice in that segment in place of an occurrence of $T_R$. "Pumping" the assembly. 

    Since the assembly is non-obstructive, we can pump forever, making the size of $\alpha$ infinite 
\end{proof}

\section*{Proof 1 ($n \times 1 $ line)}

Let $\alpha$ be an assembly of a $1 \times n$ line of tiles.

Define a TAS $\mathcal{T} = (T, \sigma, 1)$ such that $\mathcal{A}_\square[\mathcal{T}] = \{ \alpha \}$

\begin{theorem}
	$TC(\mathcal{T}) = \Omega(n)$
\end{theorem}

Proof Idea:  We will show that if there were any less than n tile types in a 1xn assembly, it would repeat forever and not terminate.

\begin{proof}
	We will go step by step to prove this theorem formally.

	\begin{enumerate}
		\item \textbf{Assume that $|T| < n$ is true to show a contradiction}

		\item \textbf{$Dup(\alpha) \neq \emptyset$}

		By the pigeonhole principle. There are $n$ tiles in $\alpha$ and less than $n$ unique tiles, so one of them must be repeated

		\item \textbf{$\alpha$ is non-obstructive}

            A straight line can only be assembled in a monotonic way. All possible binding graphs only move in a single direction away from the seed tile.

        \item \textbf{$\alpha$ is $\tau$-stable}

            Each tile has to be connected to the tiles next to it with a strength of 1. 


		\item \textbf{$\exists \alpha \in \mathcal{A}_\square[\mathcal{T}] : |\alpha| = \infty $}

            By applying the pumping lemma to $\alpha$, this assembly can be pumped indefinitely, producing an assembly of infinite length.

		\item \textbf{This is a contradiction}

		The definition of $\mathcal{T}$ states it only has a single terminal assembly, but we have found multiple.

		\item \textbf{TC($\mathcal{T}$) = $\Omega(n)$}

		Since our assumption is false, the inverse of it is true. So $|T| \geq n$.

		By the definition of a lower bound, our theorem is proved.

	\end{enumerate}
\end{proof}

\section*{Proof 2 ($n \times 2 $ rectangle)}

Let $\alpha$ be an assembly of a $2 \times n$ line of tiles.

Define a TAS $\mathcal{T} = (T, \sigma, \tau)$ such that $\mathcal{A}_\square[\mathcal{T}] = \{ \alpha \}$

\begin{theorem}
	$TC(\mathcal{T}) = \Omega(\sqrt{n})$
\end{theorem}

Proof idea: We convert the assembly to a 1xn assembly and use the lower bound from above to prove. 

\begin{proof}
		We will go step by step
	\begin{enumerate}

		\item \textbf{Assume that $|T| < \sqrt{n}$ to show a contradiction}

        \item \textbf{Replace $\mathcal{T}$ with TAS that assembles a 1xn line}

            We construct a new TAS $\mathcal{T}' = (T', \sigma', 1)$ that has $\mathcal{A}_\square[\mathcal{T}'] = \{ \alpha' \}$ where $\alpha'$ is a 1xn line.

            We build $\mathcal{T}'$ by creating $n^2$ tiles types, each tile represents a possible column of two tiles in the 2xn assembly.

            $|T'| = |T|^2$. This mean that showing a lowerbound for $\mathcal{T}'$ is equivalent to showing a lowerbound for $\mathcal{T}$.

            Showing that $TC(\mathcal{T}') = \Omega(n)$ is equivalent to showing that $TC(\mathcal{T}) = \Omega(\sqrt{n})$


		\item \textbf{$Dup(\alpha) \neq \emptyset$}

            There are n tiles and $|T|$ tile types. Since $|T| < n$, we use the pigeonhole principle to show that there must be a repeated tile type

		\item \textbf{Apply lower bound of 1xn line to show contradiction}

            The above theorem shows that $\Omega(n)$ tile types are needed to construct a 1xn line. However, we have less than $n$ tile types, so we can pump the assembly indefinitely, showing a contraction with our assumption that there is a single terminal assembly.

		\item \textbf{$TC(\mathcal{T}) = \Omega(\sqrt{n})$}

            Since assuming that we have less than $\sqrt{n}$ tile types leads to a contradiction, we can conclude that $|T| \geq \sqrt{n}$. By the definition of a lower bound, our theorem is proved.

	\end{enumerate}
\end{proof}


\section*{Proof 3 ($n \times n$ square)}

Let $\alpha$ be an assembly of a $n \times n$ tiles where all tiles have positive glue strength between them.

Define a TAS $\mathcal{T} = (T, \sigma, 1)$ such that $\mathcal{A}_\square[\mathcal{T}] = \{ \alpha \}$


\begin{theorem}
	$TC(\mathcal{T}) \geq n^2$
\end{theorem}

Proof idea: We will show that if there were any less than $n^2$ tile types in a $n \times n$ assembly, then a tile must repeat. We can build a path from the seed assembly that includes one of the duplicate tile and show that we coudl pump it by having the duplicate tile connect to something else

\begin{proof}
	We will go step by step
	\begin{enumerate}
		\item \textbf{Assume $TC(\mathcal{T}) < n^2$ for a contradiction}

		\item \textbf{$Dup(\alpha) \neq \emptyset$}

		By the pigeonhole principle, there are $n^2$ tiles in $\alpha$ and less than $n^2$ unique tiles, so one of them must be repeated

		\item \textbf{$\exists$ a connected path $\pi$ of tiles starting at the seed tile where $Dup(\pi) \neq \emptyset$}

		All tiles are connected in this assembly. We can construct this path by starting at the seed tile and going up and then over to the duplicated tile. 

		\item \textbf{$\pi$ is non obstructing}
        
            This path is monotonic since it goes up and side\-ways and is \\ non-branching since it only moves in two directions; thus, it is non-obstructive. 
        \item \textbf{$\pi$ is $\tau$-stable}

            Each tile has to be connected to the tiles next to it with a strength of 1.

        \item \textbf{$\exists \alpha \in \mathcal{A}_\square[\mathcal{T}] : |\alpha| = \infty $}

        We can now apply the pumping lemma to show that this path could be pumped to an infinite size.

        \item \textbf{This is a contradiction}

        The definition of $\mathcal{T}$ states it only has a single terminal assembly, but we have found multiple.

		\item \textbf{$TC(\mathcal{T}) \geq n^2$}

            Since there are other possible terminal assemblies, our assumption is false. There must be at least $n^2$ tile types in the assembly when they are required to all be connected. 


	\end{enumerate}
\end{proof}

\end{document}


